\Introduction

Астрономия является областью естествознания, которая изучает невероятно большую, комплексную и важную сущность — вселенную. Охватывая собой мегамир, вселенная имеет сложную структуру. Компоненты мегамира по своей природе сложны для человеческого восприятия, что обусловлено куда большими масштабами космоса, особенностями физических явлений, протекающих в нем, и прочими качественными его отличиями от макромира, непосредственной частью которого являются люди. Но, тем не менее, границы, разделяющие необъятную всеобщность на отдельные части, эфемерны, размыты, так как представляют собой продукт человеческого мышления. Доказательством этого как раз и является астрономия — продукт изучения мегамира макромиром.

В контексте изучения чего-либо зачастую встречается вопрос визуализации, и тут астрономия исключением не является. Помимо исследования, визуализация космических тел или систем может быть необходима при распространении информации о них, моделировании их поведения, использовании их образов в массовой культуре и ещё множестве других ситуаций. Компьютерная графика предоставляет наиболее простую, практичную и эффективную возможность решения этой проблемы.

В данной работе ставятся следующие задачи:

\begin{enumerate}
    \item разработать алгоритм моделирования спиральной галактики на основе физических законов;
    \item разработать алгоритм синтеза динамического изображения спиральной галактики;
    \item реализовать разработанные алгоритмы для работы в реальном времени;
    \item исследовать работу приложения.
\end{enumerate}
