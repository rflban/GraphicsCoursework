\chapter{Исследовательский раздел}
\label{cha:research}

Исследование программы проводилось на ноутбуке, подключённом к сети питания. Модель процессора ноутбука: Intel i5-8400H с максимальной тактовой частотой 2.5000 ГГц в обычном режиме и 8 логическими ядрами. Разрешение экрана: 1920$\times{}$1080.

\section{Эксперимент по замеру частоты кадров}

На рисунке \ref{img:plot} изображён график зависимости количества кадров в секунду от количества частиц в системе. Для проведения данного эксперимента программный предел частоты кадров был повышен до 60 в секунду.

\begin{figure}[H]
    \centering
    \begin{tikzpicture}
        \begin{axis}[
            width=0.8*\linewidth,
            xlabel=Количество частиц,
            ylabel=Количество кадров в секунду,
            grid=major,
            legend pos=north west,
        ]

        \addplot[color=black]
            table[x=qty,y=fps,col sep=comma]{./data/plotfps.csv};

        \end{axis}
    \end{tikzpicture}
    \caption{График зависимости количества кадров в секунду от количества частиц в системе}
    \label{img:plot}
\end{figure}

\section{Вывод}

Как показывает график, система, количество частиц в которой $\in [1000, 30000]$, успевает рисоваться практически с максимальной частотой - 60 кадров в секунду.

